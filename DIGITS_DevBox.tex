%插入样式内容
\input{Initial}
%%---------------------------------------------------------------------
\begin{document}
%%---------------------------------------------------------------------
%%---------------------------------------------------------------------
% \titlepage
\title{\vspace{-2em} DIGITS DevBox深度学习服务器\\
\normalsize{}}
\author{Dai Jialun}
\date{\vspace{-0.7em} \today \vspace{-0.7em}}
%%---------------------------------------------------------------------
\maketitle\thispagestyle{fancy}
%%---------------------------------------------------------------------
\maketitle
%\tableofcontents 
\section{硬件配置}
\begin{description}
\item[显卡] 4个ASUS(华硕)GTX 980Ti-6GD5
\definecolor{shadecolor}{rgb}{0.92,0.92,0.92}  
\begin{shaded}
\scriptsize{
芯片厂商: NVIDIA

显卡芯片: GeForce GTX 980Ti

显示芯片系列: NVIDIA GTX 900系列

核心代号: GM200

显存类型: GDDR5

显存容量: 6144MB

显存位宽: 384bit

最大分辨率: 4096×2160

接口类型: PCI Express 3.0 16X

I/O接口: HDMI接口/DVI接口/3个DisplayPort接口

电源接口: 8pin+6pin

产品尺寸:266.7×111.2×38.1mm

参考报价:5999*4=23996} 
\end{shaded}   
\item[CPU] 1个Intel(英特尔) Core i7-5960X
\definecolor{shadecolor}{rgb}{0.92,0.92,0.92}  
\begin{shaded}
\scriptsize{
CPU主频:3GHz

最高睿频:3.5GHz 

总线类型:QPI总线

总线频率:8GT/s

插槽:LGA 2011-v3 

CPU架构:Haswell

核心:八核心十六线程

制作工艺: 22纳米

功耗: 140W

三级缓存: 20MB

最大支持内存:64G

指令集:SSE4.2,AVX 2.0,AES

内存控制器:四通道:DDR4 1333/1600/2133

参考报价: 7699 RMB} 
\end{shaded}   
\item[主板] 1个ASUS(华硕)X99-E WS
\definecolor{shadecolor}{rgb}{0.92,0.92,0.92}  
\begin{shaded}
\scriptsize{
主芯片组:Intel X99

CPU插槽:LGA 2011-3

支持CPU数量: 1颗

内存类型: DDR4

内存插槽: 8*DDR4 DIMM

最大内存容量: 128GB

内存描述:支持四通道DDR4 3000(超频)/3200(O.C.)/2800(超频)/2666(超频)/2400(超频)/2133MHz内存

显卡插槽: PCI-E 3.0标准

PCI-E插槽:7×PCI-E X16显卡插槽

SATA接口: 8×SATA III接口;1×SATA Express接口;1×M.2接口(10Gb/s)

USB接口: 12×USB3.0接口(10背板+2内置);4×USB2.0接口(2背板+2内置)

版型:E-ATX板型

外形尺寸:30.5×26.7cm 

多显卡技术:支持NVIDIA 4-Way SLI四路交火技术

RAID功能:支持RAID 0,1,5,10 

尺寸:30.5 厘米 x 26.7 厘米

参考报价: 4799 RMB} 
\end{shaded}   
\item[内存] 2个CORSAIR(海盗船) VENGERNCE(复仇者)LPX 32GB(4 $\times$ 8GB) DDR4 2400MHz CMK32GX4M4A2400C14R
\definecolor{shadecolor}{rgb}{0.92,0.92,0.92}  
\begin{shaded}
\scriptsize{
内存容量:套装(4×8GB)

内存类型:DDR4

内存主频:2400MHz

参考报价:5000*2=10000 RMB} 
\end{shaded} 
\item[硬盘] 3个WesternDigital(西部数码) 4TB 7200转
\definecolor{shadecolor}{rgb}{0.92,0.92,0.92}  
\begin{shaded}
\scriptsize{
硬盘容量:4000G

缓存:64M

转速:7200rpm

接口类型:SATA3.0

接口速率:6Gb/s

参考报机:1799*3=5397 RMB} 
\end{shaded} 
\item[固态硬盘] 1个Samsung(三星)SSD 850pro 512GB
\definecolor{shadecolor}{rgb}{0.92,0.92,0.92}  
\begin{shaded}
\scriptsize{
接口类型:SATA3

硬盘尺寸:2.5英寸

参考报价:2999 RMB} 
\end{shaded} 
\item[固态硬盘] 1个Samsung SSD 512GB SM951 cache for RAID
\definecolor{shadecolor}{rgb}{0.92,0.92,0.92}  
\begin{shaded}
\scriptsize{
参考报价:3450 RMB} 
\end{shaded} 
\item[机箱] 1个CORSAIR(海盗船) 900D
\begin{shaded}
\scriptsize{
机箱样式:台式机箱(全塔)

适用主板:**EATX板型**,ATX板型,MATX板型

电源类型:标准ATX PS2电源(选配)

电源设计:下置电源

显卡限长:400mm

5.25英寸仓位:4个

3.5英寸仓位:9个

2.5英寸仓位:9个

扩展插槽:10

前置接口:4*USB 2.0;2*USB 3.0

散热性能:前:3×120mm风扇(标配),
   	顶:4×120mm或3*140mm风扇(选配),
        后:1×140mm风扇(标配),
        底:8×120mm或6*140mm风扇(选配)

尺寸:649.6×252×691.6mm

参考报价:2499 RMB } 
\end{shaded} 
\item[电源] 1个CORSAIR(海盗船) AX1500i 1500W
\definecolor{shadecolor}{rgb}{0.92,0.92,0.92}  
\begin{shaded}
\scriptsize{
功率:1500W

风扇描述:14cm风扇

电源尺寸:150x86x225mm

参考报价:3599 RMB} 
\end{shaded} 
\item[散热器] 1个CORSAIR(海盗船) H110 水冷CPU散热器
\definecolor{shadecolor}{rgb}{0.92,0.92,0.92} 
\begin{shaded}
\scriptsize{
参考报价:999 RMB} 
\end{shaded} 
\item[风扇] 6个CORSAIR(海盗船)AF120 静音版 双包装
\definecolor{shadecolor}{rgb}{0.92,0.92,0.92} 
\begin{shaded}
\scriptsize{
参考报价:89*12=1068 RMB} 
\end{shaded} 
\item[光驱] 1个AUSU(华硕)DRW-24D1ST
\definecolor{shadecolor}{rgb}{0.92,0.92,0.92} 
\begin{shaded}
\scriptsize{
参考报价:120 RMB} 
\end{shaded} 
\item[配件] 1个Thermaltake Commander FT触控式面板风扇控制器,Deepcool FAN HUB(九州风神风扇集线器)
\definecolor{shadecolor}{rgb}{0.92,0.92,0.92} 
\begin{shaded}
\scriptsize{
参考报价:299 RMB} 
\end{shaded} 
\item[显示器]
\item[键盘鼠标]
\end{description}

\section{名词解释}
\begin{description}

\item[DVI] Digital Visual Interface,数字视频接口
\begin{figure}[!h]
\centering
\includegraphics[width=0.2\textwidth]{DVI}
\end{figure}

\item[DislayPort] 高清数字显示借口标准
\begin{figure}[!ht]
\centering
\includegraphics[width=0.2\textwidth]{DisplayPort}
\end{figure}

\item[PCI-E] PCI Express,新的总线接口
\begin{figure}[!ht]
\centering
\includegraphics[width=0.2\textwidth]{PCI-E}
\end{figure}

\item[SATA Revision 3.0] Serial Advanced Technology Attachment,串行ATA规格第三版,6Gbps
\begin{figure}[!ht]
\centering
\includegraphics[width=0.2\textwidth]{SATA3}
\end{figure}

\item[SATA EXpress] SATA 3.0下一代的SATA接口,10Gbps
\begin{figure}[!ht]
\centering
\includegraphics[width=0.2\textwidth]{SATAE}
\end{figure}

\item[M.2] 一种替代MSATA新的接口规范,优势体现在速度和体积。支持Socket2和Socket3两种接口类型
\begin{figure}[!ht]
  \centering 
  \subfigure[]{ 
    \includegraphics[width=0.23\textwidth]{M2}} 
  \subfigure[]{ 
    \includegraphics[width=0.17\textwidth]{M2-MSATA}} 
  \caption{}
\end{figure}

\item[RAID] Redundant Arrays of Independent Disks,磁盘阵列。磁盘阵列是由很多价格较便宜的磁盘,组合成一个容量巨大的磁盘组,利用个别磁盘提供数据所产生加成效果提升整个磁盘系统效能。利用这项技术,将数据切割成许多区段,分别存放在各个硬盘上。
\begin{figure}[!ht]
  \centering 
  \subfigure[]{ 
    \includegraphics[width=0.24\textwidth]{RAID1}} 
  \subfigure[]{ 
    \includegraphics[width=0.15\textwidth]{RAID2}} 
  \caption{}
\end{figure}

\item[RAID5] 一种存储性能、数据安全和存储成本兼顾的存储解决方案。为系统提供数据安全保障,但保障程度要比Mirror低而磁盘空间利用率要比Mirror高。数据以块为单位分布到各个硬盘上。RAID 5不对数据进行备份,而是把数据和与其相对应的奇偶校验信息存储到组成RAID5的各个磁盘上,并且奇偶校验信息和相对应的数据分别存储于不同的磁盘上。当RAID5的一个磁盘数据损坏后,利用剩下的数据和相应的奇偶校验信息去恢复被损坏的数据。
\begin{figure}[!ht]
\centering
\includegraphics[width=0.4\textwidth]{RAID5}
\end{figure}

\item[SLI] Scalable Link Interface,可灵活伸缩的连接接口(支持多显卡技术)。这是一种可把两张或以上的显卡连在一起,作单一输出使用的技术,从而达至绘图处理效能加强的效果。
\begin{figure}[!ht]
\centering
\includegraphics[width=0.2\textwidth]{SLI}
\end{figure}

\item[DDR4] Dual Data Rate SDRAM,是一种高速CMOS动态随即访问的内存。DDR4支持2133MHz,32GB DDR4-2133达到48.4GB/s。

\item[GDDR5] Graphics Double Data Rate SDRAM version5,是一种高性能显卡用内存,需搭配支持PCI-E以上规格的显卡,高频率达4GHZ,低功耗。

\item[UEFI] Unified Extensible Firmware Interface,统一的可扩展固件接口,是一种详细描述类型接口的标准。这种接口用于操作系统自动从预启动的操作环境,加载到一种操作系统上。
\item[BIOS] Basic Input/Output System,基本输入/输出系统。
\item[固件] Firmware,固定软件(自己理解),写入EROM或EEPROM中的程序。固件担任着一个系统最基础最底层工作的软件。初期,这些硬件内所保存的程序是无法被用户直接读出或修改的,如今这些是可以重复刷写的,让固件得以修改和升级。
\item[MRB分区] MRB分区表是将磁盘的分区信息保存到磁盘的第一个扇区(MRB扇区)的64个字节中,每个分区项(文件系统、起始柱面号、磁头号等信息)占有16个字节,因此总共只能记录4个主分区,由于在一个分区项中用4个字节存储分区的总扇区数($2^{32}$),每扇区512字节($2^{9}B$),因此每个分区不能超过2TB($2^{32} \times 2^{9}B=2^{41}B=2TB$)。磁盘容量超过2TB以后,分区的起始位置也就无法表示了。
\item[GPT分区] GPT分区表是基于可扩展固件借口(EFI)使用的磁盘分区架构,支持每个磁盘可达到128个分区,且最大容量可达18EB。
\end{description}



\section{RAID5}
\subsection{RAID的优点}
\begin{itemize}
\item 可高效恢复磁盘
\item 增强了速度
\item 扩容了存储能力
\end{itemize}

\subsection{实现RAID方法}
\begin{description}
\item[硬RAID] Hardware RAID,通过用硬件(RAID卡或者磁盘阵列)来实现RAID功能。硬件RAID具备了自身的RAID控制/处理与I/O处理芯片,甚至还有阵列缓冲(Array Buffer),对CPU的占用率以及整体性能都是最优势的,但设备成本也是三最高的。Hardware RAID自成一个单元,由自身硬件和软件管理RAID,与主板和操作系统无关,即Ubuntu不需要额外的程序来管理。
\item[软RAID] Software RAID,通过用操作系统的软件程序(Linux系统下的mdadm命令)来完成RAID功能。软件RAID的所有功能都是操作系统与CPU 来完成,没有第三方的控制/处理与I/O 芯片,与主板BIOS程序无关,其效率与稳定性较低。例如在Ubuntu系统下的软RAID,其格式化、挂载、写入与重建全部由mdadm负责。
\item[伪RAID] Fake RAID,又称BIOS RAID。通过主板的集成芯片,内建RAID控制器来创建阵列,由操作系统驱动识别(主要表现在Intel Desktop的主板上表现的比较明显)。由于缺乏独立的I/O处理芯片,所以这方面的工作仍要由CPU与驱动程序来完成。另外,Fake RAID所采用的RAID控制/处理芯片的能力一般都比较弱,不能支持高的RAID等级。在Intel集成芯片的主板,主要使用Intel Rapid Storage Technology来管理,该技术主要支持Window系统,不支持Linux系统。在Linux系统下,Intel主要使用dmraid和mdadm来管理RAID,推荐使用mdadm。
\end{description}

\subsection{主板集成RAID与外插RAID卡区别}
\begin{description}
\item[性能] 主板集成的RAID,它的性能以及速度是通过主板的CPU与内存来实现的,它会占有主板一定的带宽,会影响整机的性能;外插RAID卡,有自己的CPU和内存,所以数据处理大部分都会独立处理,不会影响主板上的CPU与内存速度。总体看来,外插的RAID卡的RAID要比主板集成的RAID快得多。
\item[安全性] 主板集成的RAID,其安全性不能够得到保证,因为是通过更改主板的BIOS选项做成的,所以一旦主板损坏、主板的CMOS电池掉电或无意更改了主板BIOS的设置都会带来RAID的丢失。通过主板做成的RAID,一旦丢失,将会不能恢复,后果是非常严重的;而外插的RAID卡所做成的RAID,不会因为主板损坏、主板的CMOS电池掉电等现象对数据造成影响,所以外插的RAID卡,其安全性远远大于主板集成的。另外,Raid完全由Ubuntu的mdadm命令管理。
\end{description}


\subsection{实现RAID方法比较}
在这台DIGITS DevBox的RAID主要是Fake RAID和Software RAID,对别对应的软件是dmraid和mdadm。Intel同时支持dmraid和mdadm,但是更推荐使用mdadm。
\subsubsection{dmraid}
\begin{itemize}
\item dmraid主要是属于Fake RAID来创建、管理RAIDd的。在启动时候,由主板上的芯片驱动RAID,当载入Linux内核之后,由Linux接手管理,消耗cpu和内存等资源\footnote{http://www.cnblogs.com/linuxer/archive/2012/03/07/2441224.html}。在Ubuntu系统中,dmraid主要是将硬件的RAID映射成系统中/dev/mapper/目录下的设备,例如/dev/mapper/isw\_dfadcda\_Volume1,其中isw为intel的硬件名字,Volume1为RAID名称\footnote{http://book.51cto.com/art/200902/110754.htm}。
\item 在BIOS创建的RAID,在Ubuntu系统中,可能会出现大容量硬盘识别不正确的问题。例如,在BIOS中创建的3 个3.6TB的硬盘组成的RAID5,理论上应该为7.2TB,但是Ubuntu系统只能识别为3.6TB,容量偏小,而Ubuntu Server和Debian甚至都无法识别,不显示。
\item 不推荐使用dmraid命令。首先,dmraid从2011年已经不提供更新了,而mdadm仍然不测试和更新;其实,dmraid对于大容量硬盘的识别容易出错,如今的硬盘都是1TB以上的,对于dmraid很容易造成错误;最后,dmraid是将RAID映射成mapper,无法真正实现RAID数据恢复等高级功能。
\end{itemize}

\subsubsection{mdadm}
\begin{itemize}
\item 在linux系统中目前以MD(Multiple Devices)虚拟块设备的方式实现软件RAID,利用多个底层的块设备虚拟出一个新的虚拟设备,即使用mdadm命令\footnote{http://blog.csdn.net/yuesichiu/article/details/8502680}。
\item Fake RAID只提供廉价的控制器,RAID处理开销仍由CPU和内存负责,因此性能与效率基本与Software RAID基本一直。对于Linux系统,使用Software RAID一般比Fake RAID更稳定和安全\footnote{http://blog.163.com/jiangh\_1982/blog/static/12195052014252131760/}。
\item Ubuntu的软RAID相关命令为mdadm,其配置、测试、删除参考\footnote{http://blog.itpub.net/27771627/viewspace-1246416/}。 
\item \textbf{在没有Hardware RAID的条件下,推荐使用mdadm实现RAID}。
\end{itemize}


\subsection{创建RAID5步骤(Ubuntu下Software RAID,推荐!!!)}
在Ubuntu系统中,通常使用mdadm,即Software RAID方法来创建RAID5\footnote{http://blog.itpub.net/27771627/viewspace-1246416/}
\begin{enumerate}
\item \textbf{安装mdadm,查看实际磁盘情况}。
\begin{bash}
sudo apt-get install mdadm
sudo fdisk -l
\end{bash}
\item \textbf{初始化}。。对各个磁盘删除分区(fdisk命令),且进行格式化(mkfs命令)。小容量硬盘(不到2TB)使用MRB分区表,大容量硬盘(2TB以上)使用GPT分区\footnote{http://wangheng.org/shi-yong-parted-chuang-jian-gpt-fen-qu.html}。
\begin{bash}
sudo fdisk -l		   #查看磁盘空间以及分区
sudo fdisk /dev/sdX  #用fdisk对某块硬盘处理,/dev/sdX中X表示磁盘号,例如/dev/sdb
sudo mkfs.ext4 /dev/sdX    #用mkfs将/dev/sdX格式化为ext4格式
sudo parted /dev/sdX	#用parted工具对大容量硬盘分区,为GPT分区
\end{bash}
\item \textbf{创建RAID5}
\begin{bash}
sudo fdisk -l		   #查看磁盘空间以及分区
sudo fdisk /dev/sdX  #用fdisk对某块硬盘处理,/dev/sdX中X表示磁盘号,例如/dev/sdb
sudo mdadm -C /dev/md0 -l5 -n3 /dev/sdb1 /dev/sdb /dev/sdc /dev/sdd   #sdb,sdc和sdd为磁盘,md0为创建好的RAID盘
sudo parted /dev/sdX	#用parted工具对大容量硬盘分区,为GPT分区
\end{bash}
\item \textbf{格式化}
\begin{bash}
cat /proc/mdstat	#查看RAID恢复进度
sudo mdadm -D /dev/md0  #查看RAID详细情况
\end{bash} 
\item \textbf{挂载}
\begin{bash}
sudo mkdir /deep		#在/目录下创建/deep
sudo mount /dev/md0 /deep  	#将md0挂载到/deep下
\end{bash} 
\item \textbf{自动挂载}。将/dev/md0 /deep ext4 defaults 1 2,写入/etc/fstab。建议重启后再查看RAID的磁盘号,可能我们创建的盘号为md0,但是重启后显示为md127,如果将之前的/dev/md0直接写入/etc/fstab,如果出错,可能导致重启出现问题。
\begin{bash}
sudo vim /etc/fstab 
\end{bash}
\end{enumerate}




\section{其他工作}
\subsection{显卡驱动安装}
\subsubsection{驱动来源}
\begin{itemize}
\item 开源驱动nouveau(livecd安装时用的驱动)
\item 源(受限制驱动列表)
\item PPA源(一般是私人建的,方便群众用)
\item 自己下载编译的驱动(我们使用的方法)
\end{itemize}

\subsection{安装NVIDIA显卡驱动}
\begin{enumerate}
\item 受限制驱动列表(源)sudo apt-get install nvidia-current nvidia-settings
\item 编译驱动
	\begin{enumerate}
	\item 下载驱动 Nvidia中文官网是 http://www.nvidia.cn/page/home.html
	\item 将下载的NVIDIA-Linux-x86-185.18.14-pkg1.run驱动文件,放到 /home/用户名/ 目录下面。
	\item 编译依赖,sudo apt-get install build-essential pkg-config xserver-xorg-dev linux-headers-`uname -r`
	\end{enumerate}
\item 屏蔽开源驱动nouveau
	\begin{itemize}
	\item blacklist(推荐)
	    \begin{enumerate}
	    \item 打开终端,输入sudo vim /etc/modprobe.d/blacklist.conf
	    \item 添加 blacklist nouveau
	    \end{enumerate}
	\item grub2
	    \begin{enumerate}
	    \item 打开终端,输入sudo vim /etc/modprobe.d/blacklist.conf
	    \item 修改 GRUB\_CMDLINE\_LINUX="" 为 GRUB\_CMDLINE\_LINUX="nomodeset" 
	    \item 输入sudo update-grub
	    \end{enumerate}	
	\end{itemize}
\item 安装装备
	\begin{enumerate}
	\item 清除之前与nvidia相关的驱动程序,sudo apt-get --purge remove nvidia-*  
	\item 编译依赖,sudo apt-get install build-essential pkg-config xserver-xorg-dev linux-headers-`uname -r`
	\item 切换到虚拟终端tty1,ctl+alt+F1(如果不屏蔽nouveau,可能会出现黑屏现象);黑屏则sudo reboot,然后重启后,按下Ese或者选择low-quality,进入tty1,进行驱动的安装
	\end{enumerate}
\item 注销系统,关闭图形环境  sudo stop lightdm (Ubuntu15.04下,运行sudo systemtctl stop lightdm)
\item 安装过程 
	\begin{enumerate}
	\item 在驱动文件目录下,sudo ./NVIDIA*.run
	\end{enumerate}
\item 启动图形环境,sudo start lightdm
\end{enumerate}

\subsection{创建RAID5步骤(Fake RAID,在BISO界面)}
\begin{enumerate}
\item \textbf{确认主板芯片组是否支持RAID功能}。
\item \textbf{初始化}。对各个磁盘删除分区(fdisk命令),且进行格式化(mkfs命令)。小容量硬盘(不到2TB)使用MRB分区表,大容量硬盘(2TB以上)使用GPT分区\footnote{http://wangheng.org/shi-yong-parted-chuang-jian-gpt-fen-qu.html}。
\begin{bash}
sudo fdisk -l		   #查看磁盘空间以及分区
sudo fdisk /dev/sdX  #用fdisk对某块硬盘处理,/dev/sdX中X表示磁盘号,例如/dev/sdb
sudo mkfs.ext4 /dev/sdX    #用mkfs将/dev/sdX格式化为ext4格式
sudo parted /dev/sdX	#用parted工具对大容量硬盘分区,为GPT分区
\end{bash}
\item \textbf{在BIOS程序中设置RAID}。在Advanced Mode下,在状态栏中点击Advanced,选择PCH Storage Configuration,将SATA Controller 1 Mode Seletion设置为RAID。(只有Controller 1支持RAID模式)
\item \textbf{进入Intel Rapid Storage Technology(Intel RST)}。如果系统运行开机自检(POST)时,按下<Ctrl>+<I>进入程序界面进行管理。否则,在BIOS界面中,选择Intel Rapid Storage Technology为On后,重启再进入BIOS,在Advanced Mode选项中,在状态栏中点击Advanced,在底部可看到Intel Rapid Storage Technology的选项,点击进入设置。
\begin{figure}[!ht]
\centering
\includegraphics[width=0.4\textwidth]{raid}
\end{figure}
\item \textbf{创建RAID}。选择Create RAID Volume。
\begin{figure}[!ht]
\centering
\includegraphics[width=0.4\textwidth]{creat_raid}
\end{figure}
\item \textbf{设置RAID}。选中上图的选项中的Disks,显示下图。选择硬盘创建RAID。
\begin{figure}[!ht]
\centering
\includegraphics[width=0.4\textwidth]{select_raid}
\end{figure}
\item 创建成功的RAID,如图。RAID的status比较重要,应为Normal,如果出现Rebuild、Degrade、Failed等,请重新创建。
\begin{figure}[!ht]
\centering
\includegraphics[width=0.4\textwidth]{delete_raid}
\end{figure}
\item \textbf{格式化}。进入Ubuntu系统后,在/dev/mapper/下可看到RAID5被映射成为isw\_dfafd\_Volume1。将其格式化为Ext4文件系统。(对于大容量的硬盘的识别会出现问题,会显得比理论容量小;在Window系统下,不会出现这中情况。)
\begin{bash}
sudo mkfs.ext4 /dev/mapper/isw\_dfafd\_Volume1
\end{bash}
\item \textbf{挂载}。将格式化好的映射硬盘,挂载到/deep目录下。
\begin{bash}
sudo mkdir /deep
sudo /dev/mapper/isw\_dfafd\_Volume1 /deep
\end{bash}
\item \textbf{自动挂载}。为了重启后,直接使用映射硬盘,让其自动挂载。按照格式进入/etc/fstab
\begin{bash}
sudo vim /etc/fstab
\end{bash}
\end{enumerate}

\subsection{RAID5实验情况}
DIGITS的RAID5在各种环境下的测试,目前主板集成的RAID功能,即Intel Rapid Storage Technology,在linux下主要使用的是DM RAID和MD RAID,也就是dmraid和mdadm命令。DM RAID (dmraid),但是mdadm是比较新的应用。但是dmraid已经几年没更新了,而mdadm经过几年的测试,在工业界更受欢迎。mdadm在Window下有UI界面,在Linux下只有命令行,其产生的中间数据支持两个系统下,可用在双系统环境下。在单Linux系统下,使用mdadm比较合适。

只有dmraid的情况下,BIOS已创建RAID5
\begin{itemize}
\item Ubuntu识别/dev/mapper/isw\_dafadfadsf\_Volume1,只有3.6TB 
\item Ubuntu server无法用dmraid激活mapper,所以无法显示
\item Debian不识别
\end{itemize}

有mdadm的情况下,BIOS已创建RAID5,ubuntu系统下mdadm不创建RAID
\begin{itemize}
\item Ubuntu识别/dev/mapper/isw\_dafadfadsf\_Volume1,只有3.6TB 
\item Ubuntu server无法用dmraid激活mapper,所以无法显示
\item Debian不识别
\end{itemize}

有mdadm,BIOS不创建RAID5,Ubuntu系统下mdadm创建RAID
\begin{itemize}
\item Ubuntu不识别8TB的RAID5
\item Ubuntu server识别RAID5为8TB
\item Debian识别RAID5为8TB
\end{itemize}

因此,在现在Ubuntu14.04的系统下,决定使用mdadm创建软RAID
%%---------------------------------------------------------------------
\end{document}
