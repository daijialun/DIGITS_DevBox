%插入样式内容
\input{Initial}
%%---------------------------------------------------------------------
\begin{document}
%%---------------------------------------------------------------------
%%---------------------------------------------------------------------
% \titlepage
\title{\vspace{-2em} DIGITS\_DevBox深度学习服务器\\
\normalsize{}}
\author{Dai Jialun}
\date{\vspace{-0.7em} \today \vspace{-0.7em}}
%%---------------------------------------------------------------------
\maketitle\thispagestyle{fancy}
%%---------------------------------------------------------------------
\maketitle
%\tableofcontents 
\section{硬件配置}
\begin{description}
\item[显卡] 4个ASUS(华硕)GTX 980Ti-6GD5
\item[CPU] 1个Intel(英特尔) Core i7-5960X
\item[主板] 1个ASUS(华硕)X99-E WS
\item[内存] 2个CORSAIR(海盗船) VENGERNCE(复仇者)LPX 32GB(4 $\times$ 8GB) DDR4 2400MHz CMK32GX4M4A2400C14R
\item[硬盘] 3个WesternDigital(西部数码) 4TB 7200转
\item[固态硬盘] 1个Samsung(三星)SSD 850pro 512GB
\item[固态硬盘] 1个Samsung SSD 512GB SM951 cache for RAID
\item[机箱] 1个CORSAIR(海盗船) 900D
\item[电源] 1个CORSAIR(海盗船) AX1500i 1500W
\item[散热器] 1个CORSAIR(海盗船) H110 水冷CPU散热器
\item[风扇] 6个CORSAIR(海盗船)AF120 静音版 双包装
\item[光驱] 1个AUSU(华硕)DRW-24D1ST
\item[配件] 1个Thermaltake Commander FT触控式面板风扇控制器,Deepcool FAN HUB(九州风神风扇集线器)
\item[显示器]
\item[键盘鼠标]
\end{description}

\section{名词解释}
\begin{description}

\item[DVI] Digital Visual Interface,数字视频接口
\begin{figure}[!h]
\centering
\includegraphics[width=0.2\textwidth]{DVI}
\end{figure}

\item[DislayPort] 高清数字显示借口标准
\begin{figure}[!ht]
\centering
\includegraphics[width=0.2\textwidth]{DisplayPort}
\end{figure}

\item[PCI-E] PCI Express,新的总线接口
\begin{figure}[!ht]
\centering
\includegraphics[width=0.2\textwidth]{PCI-E}
\end{figure}

\item[SATA Revision 3.0] Serial Advanced Technology Attachment,串行ATA规格第三版,6Gbps
\begin{figure}[!ht]
\centering
\includegraphics[width=0.2\textwidth]{SATA3}
\end{figure}

\item[SATA EXpress] SATA 3.0下一代的SATA接口,10Gbps
\begin{figure}[!ht]
\centering
\includegraphics[width=0.2\textwidth]{SATAE}
\end{figure}

\item[M.2] 一种替代MSATA新的接口规范,优势体现在速度和体积。支持Socket2和Socket3两种接口类型
\begin{figure}[!ht]
  \centering 
  \subfigure[]{ 
    \includegraphics[width=0.23\textwidth]{M2}} 
  \subfigure[]{ 
    \includegraphics[width=0.17\textwidth]{M2-MSATA}} 
  \caption{}
\end{figure}

\item[RAID] Redundant Arrays of Independent Disks,磁盘阵列。磁盘阵列是由很多价格较便宜的磁盘,组合成一个容量巨大的磁盘组,利用个别磁盘提供数据所产生加成效果提升整个磁盘系统效能。利用这项技术,将数据切割成许多区段,分别存放在各个硬盘上。
\begin{figure}[!ht]
  \centering 
  \subfigure[]{ 
    \includegraphics[width=0.24\textwidth]{RAID1}} 
  \subfigure[]{ 
    \includegraphics[width=0.15\textwidth]{RAID2}} 
  \caption{}
\end{figure}

\item[RAID5] 一种存储性能、数据安全和存储成本兼顾的存储解决方案。为系统提供数据安全保障,但保障程度要比Mirror低而磁盘空间利用率要比Mirror高。数据以块为单位分布到各个硬盘上。RAID 5不对数据进行备份,而是把数据和与其相对应的奇偶校验信息存储到组成RAID5的各个磁盘上,并且奇偶校验信息和相对应的数据分别存储于不同的磁盘上。当RAID5的一个磁盘数据损坏后,利用剩下的数据和相应的奇偶校验信息去恢复被损坏的数据。
\begin{figure}[!ht]
\centering
\includegraphics[width=0.4\textwidth]{RAID5}
\end{figure}

\item[SLI] Scalable Link Interface,可灵活伸缩的连接接口(支持多显卡技术)。这是一种可把两张或以上的显卡连在一起,作单一输出使用的技术,从而达至绘图处理效能加强的效果。
\begin{figure}[!ht]
\centering
\includegraphics[width=0.2\textwidth]{SLI}
\end{figure}

\item[DDR4] Dual Data Rate SDRAM,是一种高速CMOS动态随即访问的内存。DDR4支持2133MHz,32GB DDR4-2133达到48.4GB/s。

\item[GDDR5] Graphics Double Data Rate SDRAM version5,是一种高性能显卡用内存,需搭配支持PCI-E以上规格的显卡,高频率达4GHZ,低功耗。
\end{description}

\section{软件配置名词}
\begin{description}
\item[UEFI] Unified Extensible Firmware Interface,统一的可扩展固件接口,是一种详细描述类型接口的标准。这种接口用于操作系统自动从预启动的操作环境,加载到一种操作系统上。
\item[BIOS] Basic Input/Output System,基本输入/输出系统。
\item[固件] Firmware,固定软件(自己理解),写入EROM或EEPROM中的程序。固件担任着一个系统最基础最底层工作的软件。初期,这些硬件内所保存的程序是无法被用户直接读出或修改的,如今这些是可以重复刷写的,让固件得以修改和升级。
\end{description}

\section{环境配置}
\subsection{显卡驱动安装}
\subsubsection{驱动来源}
\begin{itemize}
\item 开源驱动nouveau(livecd安装时用的驱动)
\item 源(受限制驱动列表)
\item PPA源(一般是私人建的,方便群众用)
\item 自己下载编译的驱动(我们使用的方法)
\end{itemize}

\subsection{安装NVIDIA显卡驱动}
\begin{enumerate}
\item 受限制驱动列表(源)sudo apt-get install nvidia-current nvidia-settings
\item 编译驱动
	\begin{enumerate}
	\item 下载驱动 Nvidia中文官网是 http://www.nvidia.cn/page/home.html
	\item 将下载的NVIDIA-Linux-x86-185.18.14-pkg1.run驱动文件,放到 /home/用户名/ 目录下面。
	\item 编译依赖,sudo apt-get install build-essential pkg-config xserver-xorg-dev linux-headers-`uname -r`
	\end{enumerate}
\item 屏蔽开源驱动nouveau
	\begin{itemize}
	\item blacklist(推荐)
	    \begin{enumerate}
	    \item 打开终端,输入sudo vim /etc/modprobe.d/blacklist.conf
	    \item 添加 blacklist nouveau
	    \end{enumerate}
	\item grub2
	    \begin{enumerate}
	    \item 打开终端,输入sudo vim /etc/modprobe.d/blacklist.conf
	    \item 修改 GRUB\_CMDLINE\_LINUX="" 为 GRUB\_CMDLINE\_LINUX="nomodeset" 
	    \item 输入sudo update-grub
	    \end{enumerate}	
	\end{itemize}
\item 安装装备
	\begin{enumerate}
	\item 清除之前与nvidia相关的驱动程序,sudo apt-get --purge remove nvidia-*  
	\item 编译依赖,sudo apt-get install build-essential pkg-config xserver-xorg-dev linux-headers-`uname -r`
	\item 切换到虚拟终端tty1,ctl+alt+F1(如果不屏蔽nouveau,可能会出现黑屏现象);黑屏则sudo reboot,然后重启后,按下Ese或者选择low-quality,进入tty1,进行驱动的安装
	\end{enumerate}
\item 注销系统,关闭图形环境  sudo stop lightdm (Ubuntu15.04下,运行sudo systemtctl stop lightdm)
\item 安装过程 
	\begin{enumerate}
	\item 在驱动文件目录下,sudo ./NVIDIA*.run
	\end{enumerate}
\item 启动图形环境,sudo start lightdm
\end{enumerate}

\subsection{创建RAID5}
\subsubsection{RAID的优点}
\begin{itemize}
\item 可高效恢复磁盘
\item 增强了速度
\item 扩容了存储能力
\end{itemize}

\subsubsection{RAID的分类}
\begin{description}
\item[硬RAID] hardware raid。通过用硬件来实现RAID功能的就是硬RAID,比如:各种RAID卡,还有主板集成能够做的RAID都是硬RAID。全硬的RAID 则全面具备了自己的RAID 控制/处理与I/O 处理芯片,甚至还有阵列缓冲(Array Buffer ),对CPU 的占用率以及整体性能是这三种类型中最优势的,但设备成本也是三种类型中最高的。hardRAID 自成一个单元,由自己的firmware硬件和软件,与主板和操作系统无关,即Ubuntu不需要额外的程序来管理。
\item[软RAID] software raid。通过用操作系统来完成RAID功能的就是软RAID,比如:在Linux操作系统下,用3块硬盘做的RAID5。,全软RAID 就是指RAID 的所有功能都是操作系统(OS)与CPU 来完成,没有第三方的控制/处理(业界称其为RAID 协处理器――RAID Co-Processor )与I/O 芯片。这样,有关RAID 的所有任务的处理都由CPU 来完成,可想而知这是效率最低的一种RAID 。
\item[主板RAID] 通过主板内建raid控制器创建阵列,由操作系统驱动识别。这个在Intel Desktop的主板上表现的比较明显。主要缺乏自己的I/O 处理芯片,所以这方面的工作仍要由CPU 与驱动程序来完成。而且,半软半硬RAID 所采用的RAID 控制/处理芯片的能力一般都比较弱,不能支持高的RAID 等级。FakeRaid又称BiosRaid,是由主板的Bios程序组成,与Ubuntu系统无关。但是Ubuntu提供dmraid命令与BIOS进行沟通,由FakeRAID帮助管理。基本上,ubuntu也把FakeRAID当成单一硬盘使用。
\end{description}

\subsubsection{主板集成RAID与外插RAID卡区别}
\begin{description}
\item[性能] 主板集成的RAID,它的性能以及它的速度是通过主板的CPU与内存来实现的,它会占有主板一定的带宽,会影响整机的性能;外插RAID卡,它本身由自己的CPU和内存,所以它的数据处理大部分都会由自己处理,不会影响主板上的CPU与内存速度,总体看来,外插的RAID卡的RAID要比主板集成的RAID快得多。格式化、挂载、写入与重建全部由mdadm负责。
\item[安全性] 主板集成的RAID它的安全性不能够得到保证,比如:我们用P8SCT主板做一个SATA RAID,不论你做RAID几,它是通过更改主板的BIOS选项做成的,所以一旦主板损坏、主板的CMOS电池掉电、无意更改了主板BIOS的设置都会带来RAID的丢失,通过主板做成的RAID,一旦丢失,将会不能恢复,后果是非常严重;而外插的RAID卡做成的RAID就不会因为主板损坏、主板的CMOS电池掉电等现象对数据造成影响,所以外插的RAID卡,它的安全性远远大于主板集成的。SoftRaid与主板BIOS程序无关,完全由Ubuntu的mdadm命令管理。
\end{description}


\subsubsection{创建RAID5步骤(主板BISO)}
\begin{enumerate}
\item 对各个磁盘删除分区,且进行格式化。大分区硬盘使用GPT分区,MRB分区最大只支持2TB
\item 创建,create
\item 分配,
\item 格式化 sudo mkfs.ext4 /dev/sdb
\item 挂载 sudo /dev/mapper/isw\_dfafd\_Volume1 /deep
\item 自动挂载 sudo vim /etc/fstab
\end{enumerate}

可能出现问题
\begin{itemize}
\item 在装系统过程中,在选择系统分区时,ctrl+alt+F1,输入dmraid -ay激活RAID(Ubuntu server) 这种方式其实还是属于介于软raid与硬raid之间。在启动时候,由硬件raid驱动,当载入linux内核之后,由linux接手管理,还是会消耗cpu等资源。与更传统的软件raid比较,就是启动的时候(linux内核未介入之前)系统看到的仍然是一个raid的虚拟硬盘,所以两快硬盘完全一样,要恢复重建之类的更加简单。另外的话,dmraid映射了底层的硬件raid驱硬件控制器,raid控制可能也能帮助处理一些操作,可能对性能也会有一定提高。 
\item dmraid将硬件的raid映射成/dev/mapper/下面的设备,例如/dev/mapper/isw\_dfadcda\_Volume1,其中isw为intel的硬件名字,Volume1为RAID名称。Ubuntu在安装过程中,已经将RAID显示为/dev/mapper/isw\_dfadcda\_Volume1,已经显示正确了,只不过容量出现问题,理论上的容量应该为7.2TB,但是实际情况只有3.6TB(至今未解开)。dmraid对于大容量硬盘的识别经常会出现问题(1TB识别为800G)
\item dmraid可参考http://www.cnblogs.com/linuxer/archive/2012/03/07/2441224.html
http://book.51cto.com/art/200902/110754.htm
\item Ubuntu的软RAID相关命令为mdadm,其配置、测试、删除参考http://blog.itpub.net/27771627/viewspace-1246416/
\item 目前使用的RAID为主板的Intel Rapid Storage Technology,目前驱动只支持Window,对Windows的兼容性不好,为fakeraid
\item fake raid仅提供廉价的控制器,raid处理开销仍由CPU负责,因此性能与CPU占用基本与software raid持平。 如果只有单个linux系统,使用software raid一般比fake raid更健壮,但是,在多启动环境中(例如windows与linux双系统),为了使各个系统都能正确操作相同的raid分区,就必须使用fake raid了。 http://blog.163.com/jiangh_1982/blog/static/12195052014252131760/
\end{itemize}

DIGITS的RAID5在各种环境下的测试,目前主板集成的RAID功能,即Intel@ Rapid Storage Technology,在linux下主要使用的是DM RAID和MD RAID,也就是dmraid和mdadm命令。DM RAID (dmraid),但是mdadm是比较新的应用。但是dmraid已经几年没更新了,而mdadm经过几年的测试,在工业界更受欢迎。mdadm在Window下有UI界面,在Linux下只有命令行,其产生的中间数据支持两个系统下,可用在双系统环境下。在单Linux系统下,使用mdadm比较合适。

只有dmraid的情况下,BIOS已创建RAID5
\begin{itemize}
\item Ubuntu识别/dev/mapper/isw_dafadfadsf_Volume1,只有3.6TB 
\item Ubuntu server无法用dmraid激活mapper,所以无法显示
\item Debian不识别
\end{itemize}

有mdadm的情况下,BIOS已创建RAID5,ubuntu系统下mdadm不创建RAID
\begin{itemize}
\item Ubuntu识别/dev/mapper/isw_dafadfadsf_Volume1,只有3.6TB 
\item Ubuntu server无法用dmraid激活mapper,所以无法显示
\item Debian不识别
\end{itemize}

有mdadm,BIOS不创建RAID5,Ubuntu系统下mdadm创建RAID
\begin{itemize}
\item Ubuntu不识别8TB的RAID5
\item Ubuntu server识别RAID5为8TB
\item Debian识别RAID5为8TB
\end{itemize}

因此,在现在Ubuntu14.04的系统下,决定使用mdadm创建软RAID
%%---------------------------------------------------------------------
\end{document}
